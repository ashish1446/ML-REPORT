\documentclass[12pt]{article}
\usepackage{amsmath}
\usepackage{amsthm}
\usepackage{graphicx}






\begin{document}
	
		\enlargethispage{2cm}
		
		\begin{center}
			
			\vspace*{-1cm}
			
			\textbf{\Large Introduction to Machine Learning     }(incomplete)\\[10pt]
			

\textbf{\Large Aswin}\\ [8pt]			
			
			\end{center}
		
\cleardoublepage

\section*{Overview}

I made this document as a way to learn ML for my Maters' thesis . It’s not an original work,but a compilation of scribed notes of the ML course by Dr Sanjoy Dasgupta(UCSD), [SD] which I audited.Some parts are drawn directly/inspired from Andrew NG's Stanford CS229 course notes, [ANG].My primary reference was 'The Elements of Statistical Learning' by Trevor Hastie,\\Robert Tibshirani,Jerome Friedman, [HTF].So there will be considerable overlap with the aforementioned materials.This note might lack mathematical rigor but I have tried to give proofs and supplementary topics wherever necessary.Please write to me if you find any innaccuracies.I hope this  this proves at least moderately
interesting or useful for you.

\cleardoublepage

\section*{Supervised Learning}

	In a typical scenario, we have an outcome measurement, usually quantitative (such as a stock price) or
categorical (such as heart attack/no heart attack), that we wish to predict
based on a set of features (such as diet and clinical measurements). We
have a training set of data, in which we observe the outcome and feature
measurements for a set of objects (such as people). Using this data we build
a prediction model, or learner, which will enable us to predict the outcome
for new unseen objects. A good learner is one that accurately predicts such
an outcome.
The examples above describe what is called the supervised learning problem. It is called “supervised” because of the presence of the outcome variable to guide the learning process.[HTF]

\subsection*{Variable Types and Terminology}

The outcome measurement which we wish to predict denoted as \textbf{\textit{outputs}} depend on a set of variebles denoted as \textbf{\textit{inputs}}.
Classicaly,the \textit{inputs} are independent varibales whereas \textit{outputs} are dependent variables.
The term \textit{features} will be used interchangeably with inputs.

The \textit{outputs} which we wish to predict can be qualitative or quantitative(as in blood sugar level).
When the \textit{outputs} are qualitative (as in spams or  not spams),it is referred as categorical or discrete variables and are typically represented numerically by codes,as in -spam or not spam can be coded as -1 or 1.
Depending upon the kind of output varible, the prediction task can be of two types: \textit{regression} when we predict quantitative outputs and \textit{classification} when we predict  qualitative outputs.




 The input variables/features are denoted by $x^{(i)}$ and  the space of all such $x^{(i)}$ is $X$.
 The output variable that we are trying to predict is denoted as $y^{(i)}$ and the space of all such $y^{(i)}$ is $Y$.
 A pair $(x^{(i)},y^{(i)})$ is called a training example and the dataset, we will be using to learn - a collection of $n$ training examples $ \{(x^{(i)},y^{(i)});i=1,2,...,n\} $ - is called a training set.
 The superscript $“(i)”$ in the notation is simply an index into the training set.


\cleardoublepage

\section*{Linear Regression}

Given a vector of inputs, $x^{T}=(x_{(1)},x_{(2)},..., x_{(d)})$.We need to know functions/hypotheses $h$ that can  approximate $y$ as a linear function of $x$: $$ h_{\theta}(x)=\theta_{0} + \sum_{i=1}^{d}x_{j}\theta_{j}$$
$\theta_{i}$s are parameters/weight parametrizing the space of linear functions mapping from $X$ to $Y$.The term $\theta_{0}$ is the intercept, also known as the \textit{bias} in machine learning.It is covenient to include $\theta_{0}$ in the vector of weights $\theta$  and add constant variable $1$ to the vector $x$, so that $$ h_{\theta}(x) = \theta^{T}x$$




For a training set,we have to  learn the parameters $\theta,$ so that we can predict $y$. One reasonable method seems to be to make $h(x)$ close to y,for the training examples.To formalize this, we will define a function that measures, for each value of the θ’s, how close the $h(x^{(i)})'$ s are to the corrsponding $y^{(i)}$'s.

We  define Cost/Loss function: $$J(\theta)=\frac{1}{2}\sum_{i=1}^{n}(h_{\theta}(x^{(i)})-y^{(i)})^{2},$$this is the Least Squares cost function.In this approach, we pick the coefficients $\theta$ to minimize the cost function $J$.The LS cost function is quadratic function in weights, $\theta$ and hence it's minimum always exist, but may not be unique.
Given a set of training examples $(x^{(i)},y^{(i)})$ i.e training set ,define a matrix $X$ to be the $m$-by-$n$ matrix  that contains the input values of training examples in it's rows:  

 $$ \begin{bmatrix} 
  & {(x^{(1)})^{T}} &  \\
  & \vdots & \\
  &   (x^{(m)})^{T}     &  
  \end{bmatrix} $$
 
 
Let $ \textbf y$ be the $m$-dimensional vector containing the target/output values from the training set:
 $$\begin{bmatrix} 
 & y^{(1)} &  \\
 & \vdots & \\
 &   y^{(m)}     &  
 \end{bmatrix}$$
 
 
Since $h_{\theta}(x^{(i)})=(x^{(i)})^{T}\theta$,we can verify that
 $$X \theta - \textbf y = \begin{bmatrix} 
 & {(x^{(1)})^{T}}\theta &  \\
 & \vdots & \\
 &   (x^{(m)})^{T} \theta    &  
 \end{bmatrix} - \begin{bmatrix} 
 & y^{(1)} &  \\
 & \vdots & \\
 &   y^{(m)}     &  
 \end{bmatrix}$$
 
 $$=\begin{bmatrix} 
 & h_{\theta}(x^{(1)})-y^{(1)} &  \\
 & \vdots & \\
 &  h_{\theta}(x^{(m)})- y^{(m)}     &  
 \end{bmatrix}$$
 
 
 $\frac{1}{2}(X\theta-\textbf y)^{T}(X\theta-\textbf y)=\frac{1}{2} \sum_{i=1}^{m}(h_{\theta}(x^{(i)})-y^{(i)})^{2}=J(\theta)$ This is the cost function.To minimize $J$,we have to find the derivatives with respect to $\theta$
 
 
 
 
Some matrix derivative results 
 $$\nabla_{A}tr AB = B^{T}$$ 
 $$\nabla_{A}\vert A\vert=\vert A\vert(A^{-1})^{T}$$ 
 $$\nabla_{A^{T}}f(A)= (\nabla_{A}f(A))^{T}$$ 
 $$\nabla_{A}trABA^{T}C=CAB+C^{T}AB^{T}$$
 
 
 Using the last two results 
 $$\nabla_{A^{T}}trABA^{T}C=B^{T}A^{T}C^{T}+BA^{T}C$$
 
 
 
 The cost function,  $J(\theta)=\frac{1}{2}(X\theta-\textbf{y})^{T}(X\theta-\textbf{y})$
 
 
$$ \nabla_{\theta} J(\theta)  =\nabla_{\theta}\frac{1}{2}(X\theta-\textbf{y})^{T}(X\theta-\textbf{y}) $$
 
 
 
  $$=\frac{1}{2}\nabla_{\theta}(\theta^{T}X^{T}X\theta-\theta^{T}X^{T}\textbf{y}-\textbf{y}^{T}X\theta+\textbf{y}^{T}\textbf{y}) $$
 
 
 
$$ =\frac{1}{2}\nabla_{\theta} tr(\theta^{T}X^{T}X\theta-\theta^{T}X^{T}\textbf{y}-\textbf{y}^{T}X\theta+\textbf{y}^{T}\textbf{y}) $$
 
 
 
 
$$=\frac{1}{2}\nabla_{\theta}(tr \theta^{T}X^{T}X\theta-2tr\textbf{y}^{T}X\theta)  $$
 
 

$$=\frac{1}{2}(X^{T}X\theta+X^{T}X\theta-2X^{T}\textbf{y})  =X^{T}X\theta-X^{T}\textbf{y} $$







To minimize $J,$ we set the derivative to zero and obtain:$X^{T}(X\theta-\textbf{y})=0$. If $X^{T}X$ is nonsingular then the value of $\theta $ that minimizes $J(\theta)$ is given in closed form by the equation,
$$\theta=(X^{T}X)^{-1}X^{T}\textbf{y}$$
Now that we have the parameters,we can predict the output corresponding to the input $x^{(i)}$ as $y^{(i)}=\theta^{T}x^{(i)}$





 
\end{document}